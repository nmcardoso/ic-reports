\section{Discussão e Conclusão}
Em contraste com as abordagens de aprendizagem comuns, o que fizemos neste trabalho foi construir modelos a partir dos dados de treinamento, escolher entre os melhores modelos e combiná-los. O resultado principal deste trabalho é, então, a combinação das predições de várias redes com seus respectivos melhores hiperparâmetros e a constatação de que as métricas que avaliam os resultados mostram significativa melhoria quando usamos o método de emsemble, comparado com o uso de apenas um modelo. O objetivo final era o de obter maior acurácia na classificação de galáxias do mapeamento S-PLUS, e este objetivo é atingido através do método de emsemble aqui utilizado. O produto final é a classificação de galáxias no mapeamento S-PLUS em elípticas e espirais utilizando diversas redes. A morfologia das galáxias é uma propriedade fundamental necessária, por exemplo, para estudos de formação e evolução de galáxias.

Exploramos o uso de um \emph{Ensemble}, aumentando a acurária dos resultados e diminuindo a variância das predições e proporcionando melhores resultados em relação ao que é obtido com apenas o melhor classificador indidual. Assim como o trabalho anterior \cite{bom2021}, classificamos galáxias com magnitude até $17$ mas usando um conjunto de classificadores.

Comparando este trabalho com o de \cite{bom2021}, para o limite de magnitude $17$, notamos resultados bastante similares, sendo que aquele trabalho levou em consideração imagens FITS enquanto que este trabalho utilizou imagens RGB, obtidas utilizando o \emph{software} Trilogy~\cite{coe2012clash}. Do ponto de vista computacional, muitas vezes a utilização de imagens RGB é a única opção, pois as imagens em formato FITS podem ocupar um maior espaço na memória de GPU, aumentando o tempo de processamento. Logo, foi importante mostrar aqui que as imagens RGB dão resultados similares quando o Ensemble é utilizado.

Um ponto muito importante neste trabalho é o pré-processamento. Este incluiu padronização, escalonamento e/ou normalização, como descrito na Seção \ref{section:preparacao}. O importante é que se faça o procedimento adequado para cada arquitetura. No caso específico deste trabalho, esta foi a maneira que apresentou o melhor desempenho quando combinado com a inicialização dos pesos com pré-treino na base de dados \emph{ImageNet}. Um resultado similar é mostrado no trabalho de \cite{bom2021}, em que as redes treinadas usando o pré-treino \emph{ImageNet} ultrapassaram a performance das redes treinadas sem pré-treino, mesmo com um pré-processamento de dados diferente do que foi feito aqui.

Outro ponto importante, que foi testado neste trabalho, é que, apesar de, tanto a escolha da arquitetura quanto dos hiperparâmetros da rede serem terem impacto nos resultados, em trabalhos onde apenas uma rede é utilizada, no nosso caso este impacto é minimizado pela combinação dos classificadores. Desta forma, as diferenças dos resultados causadas por variações nos hiperparâmetros de cada rede são atenuadas.

Este trabalho mostrou a melhoria dos resultados usando método de \emph{Ensemble} em relação ao projeto anterior. No futuro, este modelo pode ser usado para classificação de galáxias do levantamento S-PLUS.


\section{Trabalhos Futuros}

Durante este trabalho, foi feita uma colaboração na publicação de um artigo na revista \emph{Monthly Notices of the Royal Astronomical Society} (\url{https://doi.org/10.1093/mnras/stab1981}) sobre classificação morfológica de galáxias elípticas e espirais usando \emph{deep learnig}, para galáxias com magnitude $r_{auto} < 17$, abordando uma metodologia diferente da apresentada aqui e sem o uso de \emph{Ensemble}. Dando continuidade ao projeto, em trabalhos futuros, um outro artigo, que está em fase final de escrita, abordando o que foi apresentado neste trabalho também será publicado. Todas as análises para classificação de galáxias com magnitude $r_{auto} < 17$ apresentadas aqui serão extendidas, no artigo, para galáxias com magnitude $r_{auto} < 17.5$, possibilitando classificar galáxias com brilho mais baixo. Para estas novas galáxias, serão feitas as mesmas análises apresentadas aqui e a performance dos modelos treinados com objetos com magnitudes $r_{auto} < 17$ e $r_{auto} < 17.5$ serão comparados.