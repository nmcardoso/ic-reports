\section{Introdução}
\label{sec:intro}
Classificação morfológica é a categorização das galáxias conforme sua forma. Quando esta classificação é baseada na inspeção visual das imagens, elementos subjetivos são agregados. Em 1926, o astrônomo Edwin Hubble, na tentativa de relacionar as formas das galáxias com sua origem e evolução, criou um método hoje conhecido como \emph{Hubble Sequence} ou \emph{Tunning Fork} \cite{hubble1926, fortson2012}, que é uma tentativa de atribuir classes discretas às galáxias, de acordo com suas formas. Esta classificação, com algumas pequenas modificações e adições, ainda é usada até hoje.  No \emph{Tunning Fork}, as galáxias são classificadas como elípticas, espirais ou lenticulares, mas as formas predominantes de grandes galáxias na natureza são elípticas e espirais \cite{fortson2012}, pois acredita-se que a classe das lenticulares seja uma classe de transição. Lenticulares são muitas vezes classificadas como galáxias elípticas (mais comumente) ou espirais (menos comum). Com isto, foram criadas as classes "early-type", contendo as elípticas e lenticulares e "late-type", contendo as espirais e outras galáxias de tipo mais tardio ainda, chamadas de irregulares, que só foram incluídas no sistema de classificação muitos anos mais tarde.

O final do século 20 conheceu uma revolução na maneira de se estudar galáxias na Astronomia quando os primeiros mapeamentos de grandes áreas do céu começaram a ser feitos. O mapeamento que mais impactou a Astronomia nas últimas décadas foi o chamado SDSS\footnote{SDSS: Sloan Digital Sky Survey -- \url{https://www.sdss.org}.}.

Um programa que envolveu o SDSS e milhões de cidadãos comuns (chamado, em inglês, de projeto \emph{citizen science}) foi o chamado GalaxyZoo\footnote{\url{https://galaxyzoo.org}}, um projeto realizado em sua maioria por cidadãos sem vínculo acadêmico, que contribuíram com suas observações para a classificação de um grande número de galáxias do SDSS. A segunda liberação de dados do GalaxyZoo possui um catálogo com classificações morfológicas de trezentas mil galáxias, revisadas segundo o método de Hart et al. \cite{hart2016}. Uma subamostra destes dados, que coincide com o chamado \emph{Stripe-82}\footnote{Este é um campo equatorial do céu de 336 graus$^{2}$, que cobre a região com ascensão reta das 20:00h às 4:00h e declinação de -1,26$^{\circ}$ a +1,26$^{\circ}$}, foi utilizada neste trabalho como \emph{true table} na classificação de galáxias elípticas e espirais.

Com o avanço dos levantamentos (\emph{surveys}) digitais e conseguente aumento da quantidade de dados coletados, se torna crucial o desenvolvimento de métodos rápidos e automatizados para a classificação morfológica de galáxias sem a perda da acurácia da tradicional classificação visual \cite{yamauchi2005}. O uso de aprendizado de máquina e, mais recentemente \emph{Deep Learning} tem mostrado resultados relevantes para problemas de classificação em diversos problemas nas áreas de visão computacional e astronomia, dentre outras.

O \emph{Deep Learning} \cite{Goodfellow2016} é uma segmento específico dentro da área de aprendizado de máquina e, por conseguinte, da área de Inteligência Artificial. Consiste no desenvolvimento de redes neurais artificiais que são combinadas em um número significativamente maior do que as redes neurais tradicionais. Este tipo de técnica se transformou no estado-da-arte do reconhecimento de padrões em imagens devido a um tipo específico de rede neural conhecida como convolucional.
As redes neurais convolucionais ou CNNs da sigla em inglês \textit{Convolutional Neural Networks} \cite{lecun2015deep}, são inspiradas e propostas com certa analogia ao processamento das imagens realizadas no córtex visual de mamíferos. O processo começa quando um estímulo visual alcança a retina e equivale a um sinal que atravessa regiões específicas do cérebro. Essas regiões são responsáveis pelo reconhecimento de cada uma dessas características correspondentes \cite{karpathy2016convolutional}.
Os neurônios biológicos das primeiras regiões respondem pela identificação de formatos geométricos primários, enquanto neurônios das camadas finais têm a função de detectar características mais complexas, formadas pelas formas simples anteriormente reconhecidas \cite{karpathy2016convolutional,vedaldi2015matconvnet}. Características com padrões muito específicos do objeto são estabelecidas depois que o procedimento se repete.
De forma análoga, a CNN decompõe a tarefa de reconhecimento de um objeto em subtarefas. Para isso, durante a aprendizagem, a CNN divide a tarefa em subníveis de representação das características, posteriormente aprendendo a reconhecer novas amostras da mesma classe  \cite{lecun2015deep,vedaldi2015matconvnet}.
Desta forma, as CNNs são capazes de predizer características complexas sem a necessidade de um pré processamento e são invariantes â escala e à rotação dos dados, o que torna essencial a classificação em imagens.

Este trabalho utilizou dados do S-PLUS para classificar imagens. O S-PLUS \cite{oliveira2019} é um levantamento de galáxias do Universo Local, liderado por brasileiros, feito com um telescópio de 0.8m e com uma câmera de grande campo, localizado no Chile. A parte do mapeamento que cobre a região do chamado \emph{Stripe-82} é uma área de grande interesse dado que é coberta por diversos projetos, permitindo assim comparações e análises complementares. O S-PLUS cobriu a região com medidas de fluxo (magnitudes) em 12 bandas para três milhões de fontes (liberadas para a comunidade internacional no DR1, \cite{oliveira2019}).