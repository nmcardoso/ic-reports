\documentclass[10pt,xcolor=svgnames]{beamer}
\usepackage{palatino} %font type
\usepackage{appendixnumberbeamer}
\usefonttheme{metropolis} %Type of slides
\usefonttheme[onlymath]{serif} %font type Mathematical expressions
\usetheme[progressbar=frametitle,titleformat frame=smallcaps,numbering=counter]{metropolis} %This adds a bar at the beginning of each section.
\useoutertheme[subsection=false]{miniframes} %Circles in the top of each frame, showing the slide of each section you are at

 %enumerate each slide without counting the appendix
\setbeamercolor{progress bar}{fg=Maroon!70!Coral} %These are the colours of the progress bar. Notice that the names used are the svgnames
\setbeamercolor{title separator}{fg=DarkSalmon} %This is the line colour in the title slide
\setbeamercolor{structure}{fg=black} %Colour of the text of structure, numbers, items, blah. Not the big text.
\setbeamercolor{normal text}{fg=black!87} %Colour of normal text
\setbeamercolor{alerted text}{fg=DarkRed!60!Gainsboro} %Color of the alert box
\setbeamercolor{example text}{fg=Maroon!70!Coral} %Colour of the Example block text


\setbeamercolor{palette primary}{bg=NavyBlue!50!DarkOliveGreen, fg=white} %These are the colours of the background. Being this the main combination and so one. 
\setbeamercolor{palette secondary}{bg=NavyBlue!50!DarkOliveGreen, fg=white}
\setbeamercolor{palette tertiary}{bg=NavyBlue!40!Black, fg= white}
\setbeamercolor{section in toc}{fg=NavyBlue!40!Black} %Color of the text in the table of contents (toc)

\usepackage{amsmath,amssymb}
\usepackage{slashed}
\usepackage{cite}
\usepackage{relsize}
\usepackage{caption}
\usepackage{subcaption}
\usepackage{multicol}
\usepackage{booktabs}
\usepackage[scale=2]{ccicons}
\usepackage{pgfplots}
\usepgfplotslibrary{dateplot}
\usepackage{geometry}
\usepackage{xspace}
\usepackage[brazilian]{babel}
\usepackage[letterspace=200]{microtype}
\newcommand{\themename}{\textbf{\textsc{bluetemp}\xspace}}%metropolis}}\xspace}

\title{Classificação morfológica de galáxias usando conjunto de redes neurais convolucionais}
\author[Name]{Natanael Magalhães Cardoso$^{*}$\\ Profa. Cláudia Mendes de Oliveira$^{\dagger}$}
% \subtitle{Subtitle}
\institute[uni]{$^{*}$ Departamento de Computação e Sistemas Digitais, Escola Politécnica, USP\\$^{\dagger}$ Departamento de Astronomia, Instituto de Astronomia, Geofísica e Ciências Atmosféricas, USP}
\date{}
% \titlegraphic{\vspace{-0.5cm}\hfill\includegraphics[height=37mm]{figures/urania_color.pdf}} %You can modify the location of the logo by changing the command \vspace{}. 

\begin{document}
{
\setbeamercolor{background canvas}{bg=NavyBlue!50!DarkOliveGreen, fg=white}
\setbeamercolor{normal text}{fg=white}
\maketitle
}

\metroset{titleformat frame=smallcaps} %This changes the titles for small caps

{
  \AtBeginSection{}
  \section{Introdução}
}

\begin{frame}{Background: Astronomia}
  \begin{minipage}{0.4\textwidth}
    \begin{figure}
      \includegraphics[width=\linewidth]{fig/tunning-fork.jpg}
      \caption{Hubble's Tunning Fork}
    \end{figure}
  \end{minipage}\hfill
  \begin{minipage}{0.54\textwidth}
    \begin{figure}
      \begin{tabular}{cc}
        \includegraphics[width=0.49\linewidth,trim={50mm 65mm 35mm 55mm},clip]{papers/hubble-5.pdf} & \includegraphics[width=0.49\linewidth,trim={50mm 65mm 35mm 55mm},clip]{papers/hubble-8.pdf}
      \end{tabular}
      \caption{Extragalactic Nebulae; E. Hubble, 1926.}
    \end{figure}
  \end{minipage}
\end{frame}

\begin{frame}{Background: Neurociência}
  \begin{figure}
    \begin{tabular}{cc}
      \includegraphics[width=0.38\linewidth,trim={42mm 65mm 45mm 50mm},clip]{papers/hubel-5.pdf} & \includegraphics[width=0.38\linewidth,trim={42mm 65mm 45mm 50mm},clip]{papers/hubel-10.pdf}
    \end{tabular}
    \caption{Monkey Striate Cortex; D. H. Hubel \& T. N. Wisel, 1968.}
  \end{figure}
\end{frame}
