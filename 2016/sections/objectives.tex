\section*{Objetivos}
Desenvolver uma plataforma digital de \emph{citizen science} em biodiversidade que auxilie na coleta e divulgação de observações da fauna da Cidade Universitária Armando de Salles Oliveira. E, além disso, promover uma maior interação entre voluntários e pesquisadores, criando uma comunidade virtual de pessoas engajadas na observação e conservação dos bichos do campus.

É muito importante, também, garantir uma alta qualidade e confiabilidade nos dados adquiridos pelos voluntários para que possam ser utilizados em pesquisas da área. Por esse motivo a implementação será realizada em duas etapas: primeiramente dentro e depois fora do campus.

A plataforma digital não será implementada do zero. Ao invés disso, será escolhido algum sistema já existente para dar continuidade no desenvolvimento. Muitos desses sistesmas são implementados em outro idioma e para outras finalidades. Sendo assim, a disponibilização desses \emph{softwares} em português e de acordo com as necessidades do projeto serão as tarefas desempenhadas.

Por se tratar do desenvolvimento e implantação de um sistema já existente, o estudo de recursos exigidos, como a pilha de software\footnote{Pilha de software é um conjunto de subsistemas ou componentes de software necessário para criar uma plataforma completa. Isto inclui o sistema operacional, o sistema de banco de dados, o \emph{web server}, a \emph{framework}, dentre outros.} utilizada, serão indispensáveis para garantir a reprodutibilidade do sistema nos servidores do Laboratório de Automação Agrícola.

Além disso, como é importante que haja forte interação entre a comunidade de voluntários, a implementação de novos recursos como \emph{gamificação}\footnote{É a aplicação de elementos de jogos em contextos não relacionados aos jogos, como sistema de pontuação e recompensas.} será um dos focos de atividade.