\section{Objetivos}
Desenvolver uma plataforma digital de \emph{citizen science} em biodiversidade que auxilie na coleta e divulgação de observações da fauna da Cidade Universitária Armando de Salles Oliveira. E, além disso, promover uma maior interação entre voluntários e pesquisadores, criando uma comunidade virtual de pessoas engajadas na observação e conservação dos bichos do campus.

É muito importante, também, garantir uma alta qualidade e confiabilidade nos dados adquiridos pelos voluntários para que possam ser utilizados em pesquisas da área. Por este motivo, a implementação será realizada em duas etapas, primeiramente dentro e depois fora do campus.

A plataforma digital não será implementada do zero. Ao invés disso, será escolhido algum sistema já existente para adaptar e dar continuidade no desenvolvimento. Muitos desses sistesmas são implementados em outro idioma e para outras finalidades. Por isso, é importante disponibilizar o conteúdo do sistema em português e de acordo com as necessidades deste projeto.

E, como a interação entre a comunidade de voluntários é um dos focos do projeto, a implementação de novos recursos, como a \emph{gamificação}\footnote{Aplicação de elementos de jogos em contextos não relacionados aos jogos, como sistema de pontuação e recompensas.}, serão peças-chave para este projeto.