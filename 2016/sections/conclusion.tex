\section{Conclusão}
Na era dos grandes volumes de dados, sistemas capazes de organizá-los em informação efetiva são ferramentas essenciais. Desta forma, a utilização do código base do iNaturalist para criação de um projeto envolvendo grande volume de dados em biodiversidade, pois o iNaturalist possui uma arquitetura  robusta e escalonável, como descrita na Seção \ref{sec:arquitetura}. Recursos como a interação de usuários em estrutura de rede social são de muita importância para troca de conhecimento entre os membros da comunidade, além de proporcionar um ambiente amigável para o público geral e um maior  engajamento entre os colaboradores.

O projeto foi encerrado com uma instância do sistema desenvolvido operando em uma máquina virtual do LAA acessível via internet. No futuro, podem ser implementados recursos que melhoram a experiência de usuário, como a inclusão de nomes populares, o aprimoramento das


% • Estrutura de rede social e 'badges' na formação de uma 'reputação virtual' e influência na qualidade de dados.

% • Vantagens de uma aplicação no idioma nativo do usuário. Tanto interface quanto dados (árvore taxonômica para pesquisas utilizando nomes populares pt-br)
