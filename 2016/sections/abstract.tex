\section*{Resumo}

\emph{Citizen Science} permite que cidadãos possam usar seu potencial de observação para contribuir em diversas áreas da ciência, desde a observação de um inseto até a classificação morfológica de uma galáxia. Mas, para que estas informações sejam utilizáveis em projetos científicos, é necessário que haja uma confiabilidade dos dados. Por isso, uma plataforma digital nesta área tem grande impacto, pois, além de organizar e agilizar a coleta e visualização dos dados, permite a implantação de um processo automático e eficaz de análise da qualidade dos dados. Projetos \emph{citizen scrience} caracterizam-se pela troca de conhecimento entre os pesquisadores e os voluntários. Assim sendo, outra contribuição importante de uma plataforma digital é que a disposição dos envolvidos numa rede social facilita o engajamento dos cidadão na ciência e em outros projetos.