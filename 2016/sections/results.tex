\section*{Resultados}
Já há uma instância do iNaturalist instalada e operante no servidor do Laboratório de Automação Agrícola. No servidor estão presentes toda interface back-end e parte da interface front-end do sistema. Back-end é o conjunto de softwares responsáveis pelo armazenamento e processamento das observações enviadas. É esta interface que recebe e processa os dados enviados pelo website ou pelo aplicativo de smartphone. Front-end são os softwares que fazem interface direta com o usuário por meio de textos e imagens. São eles o portal web e o aplicativos android. Sendo que apenas o portal web é hospedado pelo servidor, o aplicativo é executado por cada smartphone.
  
Tanto a interface back-end quanto a front-end sofreram diversas customizações até que o sistema entrasse em modo operante de testes (situação atual). Nesta situação é possível fazer observações tanto pelo aplicativo móvel quanto pelo portal web e visualizá-las num mapa global de qualquer uma destas plataformas. Além disso, a estrutura de rede social da interface web permite adicionar amizades, revisar observações de outros usuários, dentre diversas outras funcionalidades.
  
Iniciou-se a customização de um projeto paralelo, utilizando o código até então desenvolvido, especializado na identificação de focos de mosquitos Aedes aegypti. Neste caso, os dados tinham um potencial caráter de alerta para possíveis surtos de doenças causadas por este vetor, já que uma observação enviada fica disponível para visualização de todos instantaneamente. Mas devido à interrupção nas atividades, este projeto foi descontinuado (ver declaração no final do relatório).
    
• Implementado: Uma versão completamente traduzida do iNaturalist em que podem ser feitas pesquisas utilizando nomes populares pt.
    
• Implementado: Estratégia de classificação da qualidade de dados por mérito utilizando 'badges' pelas atividades realizadas no iNaturalist.