\section*{Resumo}

O universo é composto de galáxias que apresentam variadas formas. Uma vez determinada a estrutura das galáxias, é possível obter informações importantes desde sua formação até sua evolução. A classificação morfológica é a catalogação de galáxias de acordo com a sua aparência visual. Ela está ligada com características físicas da galáxia. A análise visual, no entanto, introduz um viés causado pela subjetividade da observação humana. Por isso, a classificação sistemática de galáxias vem ganhando importância desde quando o astrônomo Edwin Hubble criou seu método classificação. Neste trabalho, nós combinamos classificações visuais acuradas do projeto Galaxy Zoo com métodos de \emph{Deep Learning}. Foram criados dois modelos de redes neurais convolucionais usando diferentes técnicas. O objetivo é encontrar técnicas eficientes que consigam simular a classificação visual humana, mas de forma sistematizada e automática.